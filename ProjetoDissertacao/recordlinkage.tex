%-----------------
% verso e anverso:
%-----------------
%\documentclass[10pt,twoside,openany,a4paper,portuguese]{abntex2}	
\documentclass[
	% -- opções da classe memoir --
	12pt,				% tamanho da fonte
	openany,			% capítulos começam em qualquer página
	%openright,            capítulos começam em pág ímpar (insere página vazia caso preciso)
	twoside,			% para impressão em verso e anverso. Oposto a oneside
	a4paper,			% tamanho do papel. 
	% -- opções da classe abntex2 --
	%chapter=TITLE,		% títulos de capítulos convertidos em letras maiúsculas
	%section=TITLE,		% títulos de seções convertidos em letras maiúsculas
	%subsection=TITLE,	% títulos de subseções convertidos em letras maiúsculas
	%subsubsection=TITLE,% títulos de subsubseções convertidos em letras maiúsculas
	% -- opções do pacote babel --
	brazil,				% o último idioma é o principal do documento
	]{unimontes-ppgmsc-abntex2}

% -------
% PACOTES
% -------

% --------------------
% Pacotes fundamentais 
% --------------------
\usepackage{cmap}				% Mapear caracteres especiais no PDF
\usepackage{lmodern}			% Usa a fonte Latin Modern
\usepackage[utf8]{inputenc}		% Codificacao do documento (conversão automática dos acentos)
%\usepackage[latin1]{inputenc} 	% ISO 8859-1 (windows portugues)
\usepackage[T1]{fontenc}		% Selecao de codigos de fonte.
\usepackage{tikz,fullpage}
\usepackage{indentfirst}		% Indenta o primeiro parágrafo de cada seção.
\usepackage{color}				% Controle das cores
\usepackage{graphicx}			% Inclusão de gráficos
\usepackage{pict2e}				% Pacotes graficos para figuras usando comandos de LaTeX
\usepackage{xcolor}
\usepackage{listings}           % Inclusão de arquivos de código de programação
\usepackage{showexpl}           % exibe codigos de figuras
\usepackage{verbatim}           % incluir arquivos não processados pelo Latex
\usepackage{algorithmic} %
\usepackage[portuguese,ruled,lined]{algorithm2e} % Inclusão de algoritmos(pseudo código) em portugues
\usepackage{colortbl}
\usepackage{placeins}
\usepackage{amsthm} 			% amsthm -> teorema de demonstracao
\usepackage{amsmath} 			% amsmath -> recursos avancados de matematica
\usepackage{amssymb} 			% amssymb -> symbolos e fontes adicionais (ele inclui amsfonts)
\usepackage{enumerate}			% para configurar a lista enumerada
\usepackage{longtable}			% tabela longa que quebra entre páginas
\usepackage{hhline}				% linhas duplas na tabela
%\usepackage[usenames]{color}	%Para cores ex.: {\color{Red} Texto em vermelho}



% ---

% -------------------
% Pacotes de citações
% -------------------
\usepackage[brazilian,hyperpageref]{backref}	 % Paginas com as citações na bibl
\usepackage[alf, abnt-url-package=url,]{abntex2cite}	% Citações padrão ABNT
%\usepackage{abntex2cite}	% Citações padrão ABNT

%----------------------------------------
% Definição de cores utilizadas no código
%----------------------------------------
\definecolor{dkgreen}{RGB}{0,0.6,0}
\definecolor{gray}{RGB}{0.5,0.5,0.5}
\definecolor{mauve}{RGB}{0.58,0,0.82}
\definecolor{verde}{RGB}{0.25,0.5,0.35}
\definecolor{verdecpp}{RGB}{0,0.5,0}
\definecolor{verdejava}{RGB}{0.25,0.5,0.35}
\definecolor{jpurple}{RGB}{0.5,0,0.35}
\definecolor{mygreen}{RGB}{0,0.6,0}
\definecolor{mygray}{RGB}{0.5,0.5,0.5}
\definecolor{mymauve}{RGB}{0.58,0,0.82}
\definecolor{blue}{RGB}{41,5,195}
\definecolor{graytb}{RGB}{0.8,0.8,0.8}

%-----------------------------
% configuração de codigo fonte
%-----------------------------
\lstdefinestyle{Java}
{
  language={Java},
  inputencoding=,
  basicstyle=\ttfamily\small, 
  keywordstyle=\color{jpurple}\bfseries,
  stringstyle=\color{red},
  commentstyle=\color{verdejava},
  morecomment=[s][\color{blue}]{/*}{*/},
  morecomment=[l][\color{blue}]{//},
  extendedchars=true, 
  showspaces=false, 
  showstringspaces=false, 
  extendedchars=true, 
  numbers=left,
  numberstyle=\tiny,
  breaklines=true, 
  backgroundcolor=\color{cyan!10}, 
  breakautoindent=true, 
  captionpos=b,
  xleftmargin=0pt,
  tabsize=4
}


% ------------------------ 
% CONFIGURAÇÕES DE PACOTES
% ------------------------

% -------------------------------
% Configurações do pacote backref
% Usado sem a opção hyperpageref de backref
\renewcommand{\backrefpagesname}{Citado na(s) página(s):~}
% Texto padrão antes do número das páginas
\renewcommand{\backref}{}
% Define os textos da citação
\renewcommand*{\backrefalt}[4]{
	\ifcase #1 %
		Nenhuma citação no texto.%
	\or
		Citado na página #2.%
	\else
		Citado #1 vezes nas páginas #2.%
	\fi}%


\headheight 1.4 cm
%\footheight 0.5cm
% ---


% -----------------------------------------------
% Informações de dados para CAPA e FOLHA DE ROSTO
% -----------------------------------------------
\titulo{Tratamento de dados faltantes em uma solução de record linkage com datasets heterogêneos}
\autor{Herberth Giuliano Amaral Silva}
\local{Montes Claros - MG}
\data{2015}
\instituicao{UNIVERSIDADE ESTADUAL DE MONTES CLAROS \par Centro de Ciências Exatas e Tecnológicas \par Programa de Pós Graduação Modelagem Computacional e Sistemas} 
\tipotrabalho{Projeto (Mestrado)}
\orientador[Orientador:]{Prof. Dr. Renê Rodrigues Veloso}
\preambulo{Projeto para o Programa de Pós graduação Mestrado em Modelagem Computacional e Sistemas com o objetivo de desenvolver um modelo de record linkage que suporte integração de datasets heterogêneos com dados faltantes}
% ---

%-------------------
% informações do PDF
%-------------------
\makeatletter
\hypersetup{
     	%pagebackref=true,
		pdftitle={\@title}, 
		pdfauthor={\@author},
    	pdfsubject={\imprimirpreambulo},
	    pdfcreator={record linkage},
		pdfkeywords={palavras chave}{análise}{complexidade}, 
		colorlinks=true,       		% false: boxed links; true: colored links
    	linkcolor=blue,          	% color of internal links
    	citecolor=blue,        		% color of links to bibliography
    	filecolor=magenta,      		% color of file links
		urlcolor=blue,
		bookmarksdepth=4
}

\makeatother
% --- 

% --- 
% Espaçamentos entre linhas e parágrafos 
% --- 

% O tamanho do parágrafo é dado por:
\setlength{\parindent}{1.5cm}

% Controle do espaçamento entre um parágrafo e outro:
\setlength{\parskip}{0.2cm}  % tente também \onelineskip
%\pagestyle{empty}
\pagestyle{headings}
%\pagestyle{ruledheader}

% ----------------
% compila o indice
% ----------------
\makeindex
% ---

% -------------------
% Início do documento
% -------------------
\begin{document}

% Retira espaço extra obsoleto entre as frases.
\frenchspacing 


% ----------------------------------------------------------
% ELEMENTOS PRÉ-TEXTUAIS
% ----------------------------------------------------------
% \pretextual

% ----
% Capa
% ----
\imprimircapa
% ---

% --------------
% Folha de rosto
% --------------
%\imprimirfolhaderosto
% ---

% ---
% inserir lista de ilustrações
% ---
%\pdfbookmark[0]{\listfigurename}{lof}
%\listoffigures*
%\clearpage
%\cleardoublepage
% ---

% ------------------------
% inserir lista de tabelas
% ------------------------
%\pdfbookmark[0]{\listtablename}{lot}
%\listoftables*
%\clearpage
%\cleardoublepage
% ---

% ---
% inserir lista de abreviaturas e siglas
% ---
%\begin{siglas}
%  \item[Fig.] Area of the $i^{th}$ component
%  \item[456] Isto é um número
%  \item[123] Isto é outro número
%  \item[lauro cesar] este é o meu nome
%\end{siglas}
% ---

% ---
% inserir lista de símbolos
% ---
%\begin{simbolos}
%  \item[$ \Gamma $] Letra grega Gama
%  \item[$ \Lambda $] Lambda
%  \item[$ \zeta $] Letra grega minúscula zeta
%  \item[$ \in $] Pertence
%\end{simbolos}
% ---

% -----------------
% inserir o sumario
% -----------------
\pdfbookmark[0]{\contentsname}{toc}
\tableofcontents*
\clearpage
%\cleardoublepage
% ---


% ----------------------------------------------------------
% ELEMENTOS TEXTUAIS
% ----------------------------------------------------------
% É possível usar \textual ou \mainmatter, que é a macro padrão do memoir.  
\mainmatter

% ----------------------------------------------------------
% Introdução
% Apresentação do problema, justificativa, objetivos e métodos.
% ----------------------------------------------------------
\chapter[Introdução]{Introdução}

Um dos grandes desafios da medicina moderna é a agregação de dados de pacientes em um único local (colocar citação). O desafio apresenta-se por vários fatores incluindo, mas não se limitando a: múltiplos prestadores de assistência à saúde, infraestrutura de comunicação precária, disputas políticas e problemas de sigilo médico.

Resolvidos os problemas de natureza não-tecnológica, uma solução de \textit{record linkage} faz-se necessária para unificação dos registros médicos. Tal solução é responsável por identificar tuplas em bancos de dados que correspondem à uma mesma entidade (neste caso, um paciente). Desta forma é possível identificar unicamente um paciente e seus registros médicos em uma miríade de sistemas de apoio a saúde.

No entanto, o processo de \textit{record linkage}, mesmo tendo sua fundamentação teórica consolidada há mais de 40 anos por \cite{fellegi} ainda apresenta vários desafios. Um desses desafios consiste na falta de dados (eventual ou não) necessários para o processo de descoberta de entidades únicas, algo que técnicas comuns de \textit{record linkage} não lidam bem \cite{ong}. As faltas de dados ocorrem por diversos motivos, tais como diferentes estruturas de bases de dados, diferentes requisitos de preenchimento de dados de pacientes em difierentes sitemas de apoio à saúde e negligência no preenchimento de cadastros.

Este trabalho visa buscar soluções para o problema de \textit{record linkage} em bases de dados heterogêneas (de diferentes sistemas de apoio à saúde) que possuem dados faltantes de pacientes.

% \begin{alineas}
% \item Minas a céu aberto, especificamente um estudo aprofundado sobre o problema de planejamento de minas com alocação dinâmica de veículos;
% \item Otimização multiobjetivo com ênfase em algoritmos evolucionários e VNS; 
% \item Técnicas de busca local para otimização mono e multiobjetivo. 
% \end{alineas}

\chapter{Proposta da Pesquisa}

\section{Justificativa}O processo de \textit{record linkage} é um dos desafios nas atividades de limpeza de dados (\textit{data cleasing}) em Mineração de Dados e é um processo fundamental na unificação de prontuários eletrônicos. No entanto, as técnicas clássicas de \textit{record linkage} não lidam eficientemente com dados faltantes \cite{ong}. Essa característica é uma grande desvantagem porque diferentes bases de dados podem ter diferentes características no que diz respeito à presença dados necessários no processo de \textit{record linkage}. Para não comprometer o processo de \textit{record linkage}, técnicas para lidar com falta de dados precisam ser empregadas.

\section{Objetivo Geral} Desenvolver um modelo de record linkage que suporte integração de datasets heterogêneos com dados faltantes (aleatoriamente ou não) que apresente valores mínimos de falsos-negativos;

\section{Objetivos Específicos}
\begin{alineas}
        \item Um algoritmo eficiente do ponto de vista de precisão/revocação;
        \item Uma solução de ajustes de parâmetros automatizada para o modelo em questão;
        \item Extensões de algoritmos para comparacao de tuplas em meio a dados faltantes. 
\end{alineas}


% Capitulo de textual  
% -------------------
\chapter{Referencial Teórico}
\section{\textit{Record linkage}}
\subsection{Introdução}

\textit{Record linkage} é o processo de encontrar registros duplicados em uma ou mais de dados que se referem à uma mesma entidade \cite{survey}. O processo é conhecido por deduplicação quando aplicado em um único banco de dados.

A teoria de record linkage foi desenvolvida inicialmente por \cite{fellegi}. A parte principal do processo pode ser compreendido por uma função $\mu(t_1,t_2)$ que analisa a similaridade de um par de tuplas. Esta função tem como retorno uma probabilidade, que pode ser classificada como "equivalente" (quando é possível afirmar que o par de tuplas representa a mesma entidade), "possivelmente equivalente" (quando não há informações suficientes para afirmar se é ou não equivalente) e "não equivalente" (quando há informações suficientes para afirmar que o par de tuplas \textbf{não} representam a mesma entidade).

O processo de \textit{record linkage} conta com sub-tarefas para sua realização: tais como limpeza e padronização de dados, indexação, comparação e classificação e revisão manual. Cada uma dessas sub-tarefas serão detalhadas nas subseções a seguir.

Como ilustração de funcionamento de uma solução de \textit{record linkage}, considere o conjunto de dados de exemplo abaixo:

\begin{table}[!htpb]
    \centering
    \caption{Dados demográficos}
    \label{dataset}
    \begin{tabular}{|c|c|c|c|c|}
        \hline
        \textbf{ID} & \textbf{Nome} & \textbf{Data de nascimento} & \textbf{Endereço} & \textbf{Telefone} \\ \hline
        1 & João Pedro da Silva            & 01/04/1985                                   & Rua das Camélias, 325              & 92368080                           \\ \hline
        2 & João Pedro Silva               & 01/04/1985                                   & Rua das Tulipas, 180               & 92338080                           \\ \hline
        3 & Joana Paula Silva              & 02/09/1992                                   & Rua das Tulipas, 180               & 32225478                           \\ \hline
        4 & Joana P. Silva                 & 02/09/1962                                   & Av. das Bromélias, 98              & 32225478                           \\ \hline
    \end{tabular}
\end{table}

Para deduplicar o conjunto de dados mostrado anteriormente, é necessário fazer comparações de todos os possíveis pares de tuplas ($C = n!/2(n-2)!$) utlizando a função $\mu(t_1,t_2)$. Desta forma, o conjunto gerado pela comparação dos registros previamente mencionados é:

\begin{table}[!htpb]
    \centering
    \caption{Comparações de tuplas da tabela de dados demográficos}
    \label{comparacao}
    \begin{tabular}{|c|c|c|c|c|c|c|}
        \hline
        \textbf{ID 1} & \textbf{ID 2} & \textbf{Nome} & \textbf{Data de nascimento} & \textbf{Endereço} & \textbf{Telefone} & \textbf{\textit{Score}} \\ \hline
        1 & 2 & 0.94 & 1.00 & 0.85 & 0.91 & 3.71 \\ \hline
        1 & 3 & 0.62 & 0.67 & 0.85 & 0.47 & 2.62 \\ \hline
        1 & 4 & 0.78 & 0.67 & 0.62 & 0.47 & 2.55 \\ \hline
        2 & 3 & 0.63 & 0.67 & 1.00 & 0.47 & 2.78 \\ \hline
        2 & 4 & 0.82 & 0.67 & 0.67 & 0.47 & 2.65 \\ \hline
        3 & 4 & 0.89 & 0.93 & 0.67 & 1.00 & 3.50 \\ \hline
    \end{tabular}
\end{table}

As comparações de campo-a-campo foram feitas utilizando o algoritmo de Jaro-Winkler \cite{jaro}. No final, define-se um \textit{score} que será utilizado para classificação em "equivalente", "não equivalente" e "possivelmente equivalente". Neste exemplo, os pares de tuplas $(1,2)$ e $(3,4)$ são equivalentes. Como parâmetros desta solução em particular, pode-se definir os limiares 3.0 para "possivelmente equivalentes" e 3.4 para "equivalentes", uma vez que todos os valores abaixo de 3.0 não são equivalentes e todos os valores acima de 3.4 são equivalentes.

\subsection{Limpeza e padronização dos dados}

Pelo fato que boa parte dos dados do mundo real são "sujos" e contém informações ruidosas, incompletas e mal-fortatadas, a tarefa de limpeza dos dados é crucial para o restante do processo \cite{churches}. Já é reconhecido que a falta de dados de boa qualidade pode ser um dos grandes obstáculos para uma solução de \textit{record linkage} de sucesso \cite{clark}.

O objetivo principal da tarefa de limpeza e padronização dos dados é a conversão de dados brutos de entrada em um conjunto de dados bem definido e consistente \cite{churches}.

\subsection{Indexação}

Em sua natureza, o processo de \textit{record linkage} é um processo de complexidade combinatorial, pois precisa analisar todos as combinações de pares de tuplas em uma ou mais bases de dados. Porém, a vasta maioria das comparações é feita em registros que não são equivalentes \cite{survey}, o que leva a um desperdício de recursos computacionais.

A abordagem tradicional no contexto de \textit{record linkage} para mitigar o problema previamente descrito é conhecido como \textit{blocking}. Essa estratécia consiste em dividir os dados em blocos não-sobrepostos de forma que somente dados dentro de cada bloco são comparados entre si. Um critério de \textit{blocking}, comumente chamado de chave de \textit{blocking}, é baseado em um único campo ou na concatenação de valores de vários campos \cite{survey}.

\subsection{Comparação}

Uma das fontes mais comuns de não-equivalências em tuplas de bases de dados são variações tipográficas \cite{survey2}. Desta forma, algoritmos de comparação de \textit{strings} são necessários para estipular o grau de proximidade entre campos das tuplas. Os algoritmos mais comuns são: distância de edição, distância \textit{affine gap}, distância de Smith-Waterman, distância de Jaro e distância \textit{Q-gram}.

Para complementar o trabalho dos algoritmos de comparação de strings existem algoritmos de comparação de \textit{token}, que lidam melhor com rearranjos de informação (ex: "Pedro Silva" por "Silva, Pedro" ou "Av." por "Avenida"). Alguns exemplos de algoritmos de comparação por token são: strings atômicas, \textit{WHIRL} e \textit{Q-grams} com tf.idf. Todos os métodos previamente citados são descritos com detalhes em \cite{survey2}.

\subsection{Classificação}

A comparação de strings é útil para detectar similaridade em campos únicos. No entanto, no processo de \textit{record linkage}, um único campo pode não ser o suficiente para uma boa classificação. Para esses casos, algumas abordagens complementares são necessárias. Algumas abordagens disponíveis na literatura são as abordagens probabilísticas, aprendizado supervisionado e semi-supervisionado, aprendizado ativo, distâncias, regras e não-supervisionada \cite{survey2}.


% --------------------------------------------------------------
% Capitulo com exemplos de comandos inseridos de arquivo externo 
% --------------------------------------------------------------
\chapter{Cronograma}
\begin{table}[!htpb]
	\centering
 \begin{tabular}{|c|l|c|}
 	\hline \textbf{Item} & \textbf{Descrição} & \textbf{Tempo} \\ 
 	\hline 01 & Conclusão das Disciplinas Regulares & 1º semestre 2015 \\ 
 	\hline 02 & Levantamento Bibliográfico & 1º semestre 2015 \\ 
 	\hline 03 & Definição de Conceitos &  2º trimestre 2015\\ 
 	\hline 04 & Comparação dos algoritmos & 2º semestre 2015 \\ 
 	\hline 05 & Desenvolvimento do algoritmo & 1º semestre 2016 \\ 
 	\hline 06 & Redação da Dissertação &  2º trimestre 2016\\ 
 	\hline 08 & Defesa &  3º trimestre 2016 \\ 
 	\hline 
 \end{tabular} 
	\caption{Cronograma}
	\label{t_fixa}
\end{table}
 



%-----------
% Resultados
%-----------

% ---
% Finaliza a parte no bookmark do PDF, para que se inicie o bookmark na raiz
% ---
%\bookmarksetup{startatroot}% 
% ---

% ---------
% Conclusão
% ---------
%\chapter[Considerações finais]{Considerações finais}




% ----------------------------------------------------------
% ELEMENTOS PÓS-TEXTUAIS
% ----------------------------------------------------------
\postextual

% ----------------------------------------------------------
% Referências bibliográficas
% ----------------------------------------------------------
\renewcommand{\bibname}{REFER\^ENCIAS}
\bibliography{recordlinkage}

\printindex


\end{document}
