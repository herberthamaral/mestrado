\documentclass[11pt]{article}
\usepackage{mathtools}
\usepackage{epsfig,psfrag}
\usepackage{listings}
\usepackage{color}
\usepackage{amssymb}
\usepackage[T1]{fontenc}
\usepackage{lipsum} % Package to generate dummy text throughout this template
\usepackage[brazil]{babel}
\usepackage[sc]{mathpazo} % Use the Palatino font
\usepackage[T1]{fontenc} % Use 8-bit encoding that has 256 glyphs
\usepackage[utf8]{inputenc}
\linespread{1.05} % Line spacing - Palatino needs more space between lines
\usepackage{microtype} % Slightly tweak font spacing for aesthetics
\usepackage{algpseudocode}
\usepackage{algorithmicx}

\usepackage[hmarginratio=1:1,top=32mm,columnsep=20pt]{geometry} % Document margins
\usepackage{multicol} % Used for the two-column layout of the document
\usepackage[hang, small,labelfont=bf,up,textfont=it,up]{caption} % Custom captions under/above floats in tables or figures
\usepackage{booktabs} % Horizontal rules in tables
\usepackage{float} % Required for tables and figures in the multi-column environment - they need to be placed in specific locations with the [H] (e.g. \begin{table}[H])
\usepackage{hyperref} % For hyperlinks in the PDF

\usepackage{lettrine} % The lettrine is the first enlarged letter at the beginning of the text
\usepackage{paralist} % Used for the compactitem environment which makes bullet points with less space between them

\usepackage{abstract} % Allows abstract customization
\renewcommand{\abstractnamefont}{\normalfont\bfseries} % Set the "Abstract" text to bold
\renewcommand{\abstracttextfont}{\normalfont\small\itshape} % Set the abstract itself to small italic text

\usepackage{titlesec} % Allows customization of titles
\selectlanguage{brazil}

\usepackage{fancyhdr} % Headers and footers
\pagestyle{fancy} % All pages have headers and footers
\fancyhead{} % Blank out the default header
\fancyfoot{} % Blank out the default footer
\fancyhead[C]{{\vspace{-3cm}{\hspace{-4cm}\mbox{\begin{minipage}{1.5cm} \epsfxsize=2cm
\centerline{\epsffile{unimontes.eps}}
\end{minipage}}}{\hspace{.6cm} Mineração de dados $\bullet$ 2015}}
} % Custom header text
\fancyfoot[RO,LE]{\thepage} % Custom footer text




%----------------------------------------------------------------------------------------
%    TITLE SECTION
%----------------------------------------------------------------------------------------

\title{\vspace{.5cm}\fontsize{24pt}{10pt}\selectfont\textbf{\sc Primeira Lista de Exercícios}} % Article title

\author{
\large
\textsc{Herberth Amaral}\\[2mm]
\normalsize Departamento de Ciência da Computação \\
\normalsize Universidade Estadual de Montes Claros \\
\normalsize \href{mailto:herberthamaral@gmail.com}{herberthamaral@gmail.com}
\vspace{-5mm}
}
\date{\today}

%----------------------------------------------------------------------------------------

\begin{document}

\maketitle % Insert title

\thispagestyle{fancy} % All pages have headers and footers

\newpage

\section*{Exercícios}

\begin{enumerate}
    \item{1} Objetivo
    \item{2} Detalhamento
        \begin{enumerate}
            \item{2.1}
                \begin{enumerate}
                    \item{a} Número de exemplares: 8. Número de atributos: 5.
                    \item{b} Código de aluno: inteiro positivo. Nota 1 e nota
                        2: real. Hora de estudo: inteiro positivo. Situação:
                        categórica (pode ser tratado como booleano também).
                    \item{c} Não há valores, atributos ou objetos ausentes.
                        Também não há discrepâncias ou violações de domínio.
                        Porém, como os valores das notas são próximos, podemos
                        tentar diminuir a dimensão do dataset eliminando uma
                        das notas. O atributo de código de aluno também é
                        descartável. As horas de estudo a primeira vista também
                        parecem ser um indicador de desempenho da nota. A
                        situação do aluno pode ser descartada porque é obtida
                        com uma regra básica (nota > 6).
                    \item{d} Conforme algoritmos implementados e disponíveis no
                        arquivo $lista1.py$, a correlação das notas utilizando
                        o método de Pearson é de 0.999844482844, o que torna
                        uma das notas um candidato a exclusão. Já a correlação
                        da normalização utilizando z-score de uma das notas e
                        das horas de estudo é 0.695484993417, que chega bem
                        próximo, mas não atinge o limiar proposto de > 0.7 ou <
                        -0.7.
                \end{enumerate}
            \item{2.2}
                \begin{enumerate}
                    \item{a} Número de exemplares: 12. Número de atributos: 4. Com
                        relação aos tipos: altura e largura são inteiros,
                        aspecto é real, e classe é categórica/booleana.
                    \item{b} É necessário alguns passos de pré-processamento.
                        Primeiramente há valores faltantes. Alguns desses
                        valores podem ser inferidos utilizando o fato que o
                        aspecto é a razão entre a largura e a altura.
                        Obviamente a inferência não será possível quando mais
                        de um valor dentre a altura, largura e aspecto estiver
                        faltando. Neste caso, pode-se considerar a imputação
                        desses valores. Pelos cálculos feitos e disponíveis no
                        arquivo $lista1.py$, não há grandes correlações entre
                        variáveis normalizadas.
                    \item{c} Disponível no arquivo $lista1.py$.
                \end{enumerate}
        \end{enumerate}
\end{enumerate}

\end{document}
