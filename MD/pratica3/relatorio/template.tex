\documentclass[11pt]{article}
\usepackage{array}
\usepackage{mathtools}
\usepackage{epsfig,psfrag}
\usepackage{listings}
\usepackage{color}
\usepackage{amssymb}
\usepackage[T1]{fontenc}
\usepackage{lipsum} % Package to generate dummy text throughout this template
\usepackage[brazil]{babel}
\usepackage[sc]{mathpazo} % Use the Palatino font
\usepackage[T1]{fontenc} % Use 8-bit encoding that has 256 glyphs
\usepackage[utf8]{inputenc}
\linespread{1.05} % Line spacing - Palatino needs more space between lines
\usepackage{microtype} % Slightly tweak font spacing for aesthetics
\usepackage{algpseudocode}
\usepackage{algorithmicx}

\usepackage[hmarginratio=1:1,top=32mm,columnsep=20pt]{geometry} % Document margins
\usepackage{multicol} % Used for the two-column layout of the document
\usepackage[hang, small,labelfont=bf,up,textfont=it,up]{caption} % Custom captions under/above floats in tables or figures
\usepackage{booktabs} % Horizontal rules in tables
\usepackage{float} % Required for tables and figures in the multi-column environment - they need to be placed in specific locations with the [H] (e.g. \begin{table}[H])
\usepackage{hyperref} % For hyperlinks in the PDF

\usepackage{lettrine} % The lettrine is the first enlarged letter at the beginning of the text
\usepackage{paralist} % Used for the compactitem environment which makes bullet points with less space between them

\usepackage{abstract} % Allows abstract customization
\renewcommand{\abstractnamefont}{\normalfont\bfseries} % Set the "Abstract" text to bold
\renewcommand{\abstracttextfont}{\normalfont\small\itshape} % Set the abstract itself to small italic text

\usepackage{titlesec} % Allows customization of titles
\selectlanguage{brazil}

\usepackage{fancyhdr} % Headers and footers
\pagestyle{fancy} % All pages have headers and footers
\fancyhead{} % Blank out the default header
\fancyfoot{} % Blank out the default footer
\fancyhead[C]{{\vspace{-3cm}{\hspace{-4cm}\mbox{\begin{minipage}{1.5cm} \epsfxsize=2cm
\centerline{\epsffile{unimontes.eps}}
\end{minipage}}}{\hspace{.6cm} Mineração de dados $\bullet$ 2015}}
} % Custom header text
\fancyfoot[RO,LE]{\thepage} % Custom footer text




%----------------------------------------------------------------------------------------
%    TITLE SECTION
%----------------------------------------------------------------------------------------

\title{\vspace{.5cm}\fontsize{24pt}{10pt}\selectfont\textbf{\sc Terceiro trabalho prático}} % Article title

\author{
\large
\textsc{Herberth Amaral e Marcelino Macedo}\\[2mm]
\normalsize Departamento de Ciência da Computação \\
\normalsize Universidade Estadual de Montes Claros \\
\normalsize Professor Dr. Renato Dourado Maia\\
\vspace{-5mm}
}
\date{\today}

%----------------------------------------------------------------------------------------

\begin{document}

\maketitle % Insert title

\thispagestyle{fancy} % All pages have headers and footers

\newpage

\section{Introdução}
\section{Objetivos}

Os objetivos deste trabalho são:

\begin{enumerate}
    \item Explorar opções de algoritmos de classificação;
    \item Analisar o desempenho desses algoritmos segundo critérios de precisão, tempo de treinamento e de execução.
\end{enumerate}

\section{Metodologia}

Falar que usamos validação cruzada com 10-fold.

\section{Desenvolvimento}

O presente trabalho foi desenvolvido na linguagem Python 2 e seu respectivo
código (com as devidas instruções de execução) pode ser encontrado no endereço
\url{https://github.com/herberthamaral/mestrado/tree/master/MD/pratica3}.

Ferramentas auxiliares foram utilizadas no desenvolvimento deste trabalho:

\begin{enumerate}
    \item Numpy;
    \item Scikit-learn (colocar referências);
\end{enumerate}

Todos os testes foram executados em um Intel Core i5 de segunda geração (2
processadores, 4 \textit{threads}) com 6GB de RAM.

Os seguintes algoritmos foram avaliados:

\begin{enumerate}
    \item \textit{sklearn.linear\_model.SGDClassifier};
    \item \textit{sklearn.linear\_model.Perceptron};
    \item \textit{sklearn.linear\_model.PassiveAggressiveClassifier};
    \item \textit{sklearn.lda.LDA};
    \item \textit{sklearn.kernel\_ridge.KernelRidge};
    \item \textit{sklearn.svm.SVC};
    \item \textit{sklearn.svm.NuSVC};
    \item \textit{sklearn.svm.LinearSVC};
    \item \textit{sklearn.linear\_model.SGDClassifier};
    \item \textit{sklearn.neighbors.RadiusNeighborsClassifier};
    \item \textit{sklearn.neighbors.KNeighborsClassifier};
    \item \textit{sklearn.naive\_bayes.GaussianNB};
    \item \textit{sklearn.naive\_bayes.MultinomialNB};
    \item \textit{sklearn.naive\_bayes.BernoulliNB};
    \item \textit{sklearn.tree.DecisionTreeClassifier};
    \item \textit{sklearn.ensemble.GradientBoostingClassifier};
\end{enumerate}

\subsection{Técnicas de implementação}

Aproveitando do fato que as classes que implementam os algoritmos de
classificação seguem a mesma interface, implementamos o algoritmo de testes dos
classificadores utilizando técnicas de reflexão. Essas técnicas permitiram que o
algoritmo de teste ficasse mais generalista e resumido, uma vez que não é
necessário implementar um teste específico para cada algoritmo.

Devido ao alto tempo de execução e a alta possibilidade de paralelismo,
utilizamos o módulo de multiprocessamento do Python para diminuir o tempo de
execução dos testes e fazer melhor uso dos recursos computacionais. A
quantidade de subprocessos é determinada pela quantidade de processadores
disponíveis no ambiente que o algoritmo é executado (4 subprocessos utilizando
a máquina de testes descrita anteriormente).

Além da execução paralela, duas pequenas otimizações foram feitas com o intuito
de dimiuir o tempo de processamento. Duas técnicas foram utilizadas:
\textit{lazy load} dos datasets e uma otimização do algoritmo \textit{minmax}
em que reduzimos a complexidade de $O(n^2)$ para $O(n)$.


Pelo mesmo motivo de tempo de execução apontado anteriormente, implementamos um
mecanismo de retomada da execução do algoritmo de testes: a cada uma das cem
iterações salvamos o estado da execução. O algoritmo continua a execução de
onde parou caso uma parada aconteça.

\subsection{Pré-processamento de dados}

Com exceção da base de dados \textit{banknot}, as bases de dados contêm
atributos não-numéricos que precisam ser tratados antes. Esses atributos foram
substituídos por valores inteiros com o intuito de permitir o uso nos
classificadores. Esse pré-processamento pode ser analisado no arquivo
\textit{main.py} na função \textit{trata\_datasets()}.

Além disso, todos os dados numéricos (exceto as classes) foram normalizados
utilizando a técnica \textbf{minmax} ($minmax(X) = {x-Min(X)\over{Max(X)-Min(X)}}, \forall x \in X)$. 
O minmax normaliza os dados no intervalo $[0,1]$ e isso é especialmente
interessante para os classificadores Bayesianos, os quais não aceitam entradas
negativas.

\subsection{Execução e pós-processamento}

Com 16 algoritmos de classificação, três datasets e validação cruzada 10-fold,
cada iteração demora cerca de seis minutos com a configuração detalhada
anteriormente. Como executamos cem iterações, a execução total do nosso
algoritmo ficou em cerca de dez horas.

A implementação de recursos de paralelismo ajudou a diminuir o tempo de
execução, porém, notamos que boa parte do tempo é gasta sintetizando os
resultados do teste, o que usa apenas um processador. Fazendo uma analogia com
o modelo de programação MapReduce (colocar referência), uma quantidade
significativa de tempo foi gasta na fase de redução, porém uma boa fatia de
tempo foi economizada na fase de mapeamento.

Salvamos um arquivo com os dados produzidos pelo nosso algoritmo no formato
JSON a cada iteração. Esse arquivo contém as seguintes informações:

\begin{enumerate}
    \item Tamanho do dataset;
    \item Tempo (em segundos) usado para treinamento;
    \item Tempo (em segundos) usado para validação;
    \item Precisão (no intervalo $[0..1]$) do modelo;
    \item Quantidade de erros de validação;
    \item \textit{Datset} utilizado;
    \item Algoritmo de classificação utilizado;
    \item Matriz de confusão;
\end{enumerate}

Para facilitar a análise dos dados, desenvolvemos um utlitário para unificar os
arquivos JSON e converte-los para CSV (com exceção da matriz de confusão).

\section{Resultados}
\section{Considerações finais}
\section{Referências}
\section{Anexos}
\subsection{Matrizes de confusão}


\end{document}
