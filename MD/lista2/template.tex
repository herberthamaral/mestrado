\documentclass[11pt]{article}
\usepackage{mathtools}
\usepackage{epsfig,psfrag}
\usepackage{listings}
\usepackage{color}
\usepackage{amssymb}
\usepackage[T1]{fontenc}
\usepackage{lipsum} % Package to generate dummy text throughout this template
\usepackage[brazil]{babel}
\usepackage[sc]{mathpazo} % Use the Palatino font
\usepackage[T1]{fontenc} % Use 8-bit encoding that has 256 glyphs
\usepackage[utf8]{inputenc}
\linespread{1.05} % Line spacing - Palatino needs more space between lines
\usepackage{microtype} % Slightly tweak font spacing for aesthetics
\usepackage{algpseudocode}
\usepackage{algorithmicx}

\usepackage[hmarginratio=1:1,top=32mm,columnsep=20pt]{geometry} % Document margins
\usepackage{multicol} % Used for the two-column layout of the document
\usepackage[hang, small,labelfont=bf,up,textfont=it,up]{caption} % Custom captions under/above floats in tables or figures
\usepackage{booktabs} % Horizontal rules in tables
\usepackage{float} % Required for tables and figures in the multi-column environment - they need to be placed in specific locations with the [H] (e.g. \begin{table}[H])
\usepackage{hyperref} % For hyperlinks in the PDF

\usepackage{lettrine} % The lettrine is the first enlarged letter at the beginning of the text
\usepackage{paralist} % Used for the compactitem environment which makes bullet points with less space between them

\usepackage{abstract} % Allows abstract customization
\renewcommand{\abstractnamefont}{\normalfont\bfseries} % Set the "Abstract" text to bold
\renewcommand{\abstracttextfont}{\normalfont\small\itshape} % Set the abstract itself to small italic text

\usepackage{titlesec} % Allows customization of titles
\selectlanguage{brazil}

\usepackage{fancyhdr} % Headers and footers
\pagestyle{fancy} % All pages have headers and footers
\fancyhead{} % Blank out the default header
\fancyfoot{} % Blank out the default footer
\fancyhead[C]{{\vspace{-3cm}{\hspace{-4cm}\mbox{\begin{minipage}{1.5cm} \epsfxsize=2cm
\centerline{\epsffile{unimontes.eps}}
\end{minipage}}}{\hspace{.6cm} Mineração de dados $\bullet$ 2015}}
} % Custom header text
\fancyfoot[RO,LE]{\thepage} % Custom footer text




%----------------------------------------------------------------------------------------
%    TITLE SECTION
%----------------------------------------------------------------------------------------

\title{\vspace{.5cm}\fontsize{24pt}{10pt}\selectfont\textbf{\sc Segunda Lista de Exercícios}} % Article title

\author{
\large
\textsc{Herberth Amaral}\\[2mm]
\normalsize Departamento de Ciência da Computação \\
\normalsize Universidade Estadual de Montes Claros \\
\normalsize \href{mailto:herberthamaral@gmail.com}{herberthamaral@gmail.com}
\vspace{-5mm}
}
\date{\today}

%----------------------------------------------------------------------------------------

\begin{document}

\maketitle % Insert title

\thispagestyle{fancy} % All pages have headers and footers

\newpage

\section*{Exercícios}

\begin{enumerate}
    \item{1} Objetivo
    \item{2} Detalhamento
        \begin{enumerate}
            \item{2.1}
                \begin{enumerate}
                    \item{a} $G_1 = \{x_1,x_2,x_3\}; G_2 = \{x_1,x_2,x_3\}$
                    \item{b} $C_1 = [2.00, 1.33], C_2=[5.00, 2.66]$
                    \item{c} Testando com dois tipos de normalização (escore-z e minmax), nenhuma
técnica fez com que o algoritmo de agrupamento apresentasse resultados
diferentes. 

Foram feitos testes com apenas uma iteração e com três iterações, com dados
normalizados e não-normalizados, sempre com os mesmos resultados. Todos os
testes foram executados  utilizando $C1 = (1,2)$ e $C2 = (4,2)$ como centróides
iniciais.

No entanto, por causa de discrepâncias nas escalas dados reais, torna-se
necessário normalizar os dados antes dos agrupamentos. Especula-se que a
não-diferença notada no dataset deste exemplo deve-se ao fato dos dois tipos de
dados estarem na mesma escala.
                    \item{d} Executando o algoritmo com dois critérios de parada (100 iterações e 0.5
para movimentação de centróide), em todos os casos o critério de parada de
movimentação de centróide foi o motivo de parada. Portanto pode-se dizer que o
algoritmo estabilizou.
                    \item{e} Uma possível interpretação é a classe social do cliente. Clientes com
melhor condição financeira gastam mais com vestuário e tendem a gastar pelo menos
a mesma quantia com alimentação. 
                \end{enumerate}
            \item{2.2}
                \begin{enumerate}
                    \item{a}
                    \begin{verbatim}
x2--
    |-- 1.41
x1--   | ------- 1.42
x3-----         |
x4 ---          | * - 3.20 
      | -- 1.00 |
x5 ---    | ---- 2.41
x6 -------
                    \end{verbatim}
                    \item{b} O intervalo de corte fica entre os grupos $\{x1, x2, x3\}$ e $\{x4, x5, x6\}$, conforme indicado com $*$ na questão anterior.
                \end{enumerate}
        \end{enumerate}
\end{enumerate}

\end{document}
