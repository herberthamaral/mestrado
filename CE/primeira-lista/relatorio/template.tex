\documentclass[11pt]{article}
\usepackage{mathtools}
\usepackage{epsfig,psfrag}
\usepackage{calligra}
\usepackage{listings}
\usepackage{color}
\usepackage{amssymb}
\usepackage[T1]{fontenc}
\usepackage[brazil]{babel}
\usepackage[sc]{mathpazo} % Use the Palatino font
\usepackage[T1]{fontenc}
\usepackage[utf8]{inputenc}
\linespread{1.05} % Line spacing - Palatino needs more space between lines
\usepackage{microtype} % Slightly tweak font spacing for aesthetics
\usepackage{algpseudocode}
\usepackage{algorithmicx}

\usepackage[hmarginratio=1:1,top=32mm,columnsep=20pt]{geometry} % Document margins
\usepackage{multicol} % Used for the two-column layout of the document
\usepackage[hang, small,labelfont=bf,up,textfont=it,up]{caption} % Custom captions under/above floats in tables or figures
\usepackage{booktabs} % Horizontal rules in tables
\usepackage{float} % Required for tables and figures in the multi-column environment - they need to be placed in specific locations with the [H] (e.g. \begin{table}[H])
\usepackage{hyperref} % For hyperlinks in the PDF

\usepackage{lettrine} % The lettrine is the first enlarged letter at the beginning of the text
\usepackage{paralist} % Used for the compactitem environment which makes bullet points with less space between them

\usepackage{abstract} % Allows abstract customization
\renewcommand{\abstractnamefont}{\normalfont\bfseries} % Set the "Abstract" text to bold
\renewcommand{\abstracttextfont}{\normalfont\small\itshape} % Set the abstract itself to small italic text

\usepackage{titlesec} % Allows customization of titles
\selectlanguage{brazil}

\usepackage{fancyhdr} % Headers and footers
\pagestyle{fancy} % All pages have headers and footers
\fancyhead{} % Blank out the default header
\fancyfoot{} % Blank out the default footer
\fancyhead[C]{{\vspace{-3cm}{\hspace{-4cm}\mbox{\begin{minipage}{1.5cm} \epsfxsize=2cm
\centerline{\epsffile{unimontes.eps}}
\end{minipage}}}{\hspace{.6cm} Computação evolutiva $\bullet$ 2015}}
} % Custom header text
\fancyfoot[RO,LE]{\thepage} % Custom footer text




%----------------------------------------------------------------------------------------
%    TITLE SECTION
%----------------------------------------------------------------------------------------

\title{\vspace{.5cm}\fontsize{24pt}{10pt}\selectfont\textbf{\sc Primeira lista de exercícios}} % Article title

\author{
\large
\textsc{Herberth Amaral}\\[2mm]
\normalsize Departamento de Ciência da Computação \\
\normalsize Universidade Estadual de Montes Claros \\
\normalsize \href{mailto:herberthamaral@gmail.com}{herberthamaral@gmail.com}
\vspace{-5mm}
}
\date{\today}

%----------------------------------------------------------------------------------------

\begin{document}

\maketitle % Insert title

\thispagestyle{fancy} % All pages have headers and footers

\section{Respostas}

Os testes foram feitos utilizando o teste da soma de postos de Wilcoxon, em que
a hipótese nula diz que as duas amostras foram tiradas da mesma distribuição e
que a hipótese alternativa é que os valores de uma amostra tem uma maior chance
de de serem maiores do que a outra amostra.

Uma vez comprovada a hipótese alternativa com intervalo de confiança de 5\% (ou
seja, p < 0.05), um teste simples de mediana determinará a melhor alternativa.

Cada teste conta com dados de 60 execuções e para maior facilidade de
reprodução, a semente geradora de números aleatórios foi inicializada com um
valor fixo.

\subsection{Primeiro teste}

O primeiro teste mostra que o ponto de corte uniforme foi melhor
estatisticamente (p=0.069), com mediana 21 contra 20. Os demais resultados do
teste foram: Maior/Menor/Media/Desvio/Sucessos:
25/17/20.63/1.67/0. Q1/Q2/Q3/ 19.0/21.0/22.0

\subsection{Segundo teste}

O segundo teste tem o intuito de descobrir o melhor operador de seleção: roleta ou torneio.

Com resultado expressivamente melhor, a seleção por roleta se mostrou melhor
(p=4.99e-20). Os demais dados do teste foram:
Maior/Menor/Media/Desvio/Sucessos: 27/25/26.8/0.439696865276/49. Q1/Q2/Q3/
27.0/27.0/27.0

\subsection{Terceiro teste}

O terceiro teste tem o intuito de avaliar os operadores de mutação: bit-a-bit ou escolha aleatória do bit.

Por uma longa margem (7 pontos na mediana) o operdor bit-a-bit superou o operador de escolha aleatória (p=3.46e-21).

Os demais resultados do teste foram: Maior/Menor/Media/Desvio/Sucessos: 27/26/26.9166666667/0.276385399196/55. Q1/Q2/Q3/ 27.0/27.0/27.0.

É um número bem expressivo de sucessos considerando que ainda não há elitismo.

\subsection{Quarto teste}

O quarto teste tem o intuito de avaliar a probabilidade de cruzamento. Este
teste é diferente dos demais porque não se trata de um teste pareado e sim de
um teste com 4 amostras diferentes. Por conta disso, os resultados serão
apresentados de forma diferente: ao invés de uma, três comparações (20\% com
50\%, 20\% com 80\% e 50\% com 80\%).

Os resultados foram os seguintes: 50\% foi melhor que 20\% (p=1.69e-10), 80\%
foi melhor que 20\% (p=5.87e-14) e 80\% foi melhor que 50\% por uma pequena
margem (p=0.03).

Os resultados com 80\% são os mesmos do terceiro teste.

\subsection{Quinto teste}

O quinto teste tem o intuito de averiguar o efeito da troca da probabilidade de
mutação. São 5 opções de probabilidade de mutação, então 10 testes são
necessários.

O teste de Wilcoxon mostrou que há uma diferença pequena entre as duas melhores
taxas - de 2.5\% e de 5\%  (p=0.01), mas a mediana das duas possui o mesmo
valor (27).

\subsection{Sexto teste}

O sexto teste tem o intuito de avaliar as estratégias elitistas versus as não-elitistas.

Supreendentemente, não houve diferenças significativas entre estratégias
elitistas versus as não-elitistas (p=0.86). Provavelmente, a boa escolha dos
demais operadores fez com que as estratégias elitistas vs não elitistas não
fizessem diferença.

Para validar a hipótese feita no parágrafo anterior, trocou-se o operador de
seleção (roleta para torneio) com o intuito de piorar o algoritmo e analisar as
diferenças entre elistimo vs. não-elitismo.

Desta vez os resultados penderam muito para o lado do elitismo (p=2.46e-21),
com uma diferença de 3 pontos na mediana dos valores de fitness dos resultados.
É importante notar que mesmo com uma configuração ruim de um operador, todas as
execuções chegaram no mínimo global.

\subsection{Avaliação final}

Com base nos seis testes anteriores, a melhor configuração para o AG é:

\begin{itemize}
    \item \textbf{Cruzamento:} com ponto de corte uniforme;
    \item \textbf{Seleção:} com roleta;
    \item \textbf{Mutação:} bit-a-bit;
    \item \textbf{Probabilidade de cruzamento:} 80\%;
    \item \textbf{Probabilidade de mutação:} 5\% ou 2.5\%;
\end{itemize}

\subsubsection{Teste com números de gerações - 20, 25, 30, 50}

Não houve diferenças significativas com a variação dos números de gerações
utilizando as configurações ótimas. Portanto, podemos usar o menor número de
gerações (25).

Os testes mostram que a diferença passa a não ser significativa a partir de 20 gerações.

\subsection{Conclusão}

A melhor configuração para o AG segundo os testes realizados anteriormente, são:

\begin{itemize}
    \item \textbf{Cruzamento:} com ponto de corte uniforme;
    \item \textbf{Seleção:} com roleta;
    \item \textbf{Mutação:} bit-a-bit;
    \item \textbf{Probabilidade de cruzamento:} 80\%;
    \item \textbf{Probabilidade de mutação:} 5\% ou 2.5\%;
    \item \textbf{Número de gerações:} a partir de 25;
    \item \textbf{Número de indivíduos:} a partir de 20;
\end{itemize}

%Os scripts utilizados nesses testes encontram-se disponíveis em \url{

\end{document}
