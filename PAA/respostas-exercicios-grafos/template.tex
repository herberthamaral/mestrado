\documentclass[11pt]{article}
\usepackage{epsfig,psfrag}
\usepackage{amssymb}
\usepackage[T1]{fontenc}
\usepackage{lipsum} % Package to generate dummy text throughout this template
\usepackage[brazil]{babel}
\usepackage[sc]{mathpazo} % Use the Palatino font
\usepackage[T1]{fontenc} % Use 8-bit encoding that has 256 glyphs
\usepackage[utf8]{inputenc}
\linespread{1.05} % Line spacing - Palatino needs more space between lines
\usepackage{microtype} % Slightly tweak font spacing for aesthetics
\usepackage{algpseudocode}
\usepackage{algorithmicx}

\usepackage[hmarginratio=1:1,top=32mm,columnsep=20pt]{geometry} % Document margins
\usepackage{multicol} % Used for the two-column layout of the document
\usepackage[hang, small,labelfont=bf,up,textfont=it,up]{caption} % Custom captions under/above floats in tables or figures
\usepackage{booktabs} % Horizontal rules in tables
\usepackage{float} % Required for tables and figures in the multi-column environment - they need to be placed in specific locations with the [H] (e.g. \begin{table}[H])
\usepackage{hyperref} % For hyperlinks in the PDF

\usepackage{lettrine} % The lettrine is the first enlarged letter at the beginning of the text
\usepackage{paralist} % Used for the compactitem environment which makes bullet points with less space between them

\usepackage{abstract} % Allows abstract customization
\renewcommand{\abstractnamefont}{\normalfont\bfseries} % Set the "Abstract" text to bold
\renewcommand{\abstracttextfont}{\normalfont\small\itshape} % Set the abstract itself to small italic text

\usepackage{titlesec} % Allows customization of titles
\selectlanguage{brazil}

\usepackage{fancyhdr} % Headers and footers
\pagestyle{fancy} % All pages have headers and footers
\fancyhead{} % Blank out the default header
\fancyfoot{} % Blank out the default footer
\fancyhead[C]{{\vspace{-3cm}{\hspace{-4cm}\mbox{\begin{minipage}{1.5cm} \epsfxsize=2cm
\centerline{\epsffile{unimontes.eps}}
\end{minipage}}}{\hspace{.6cm} Projeto e Análise de Algoritmos $\bullet$ 2015}} 
} % Custom header text
\fancyfoot[RO,LE]{\thepage} % Custom footer text




%----------------------------------------------------------------------------------------
%	TITLE SECTION
%----------------------------------------------------------------------------------------

\title{\vspace{.5cm}\fontsize{24pt}{10pt}\selectfont\textbf{\sc Lista de Exercícios - Grafos}} % Article title

\author{
\large
\textsc{Herberth Amaral}\\[2mm] % Your name
\normalsize Departamento de Ciência da Computação \\
\normalsize Universidade Estadual de Montes Claros \\
\normalsize \href{mailto:herberthamaral@gmail.com}{herberthamaral@gmail.com} % Your email address
\vspace{-5mm}
}
\date{\today}

%----------------------------------------------------------------------------------------

\begin{document}

\maketitle % Insert title

\thispagestyle{fancy} % All pages have headers and footers

%----------------------------------------------------------------------------------------
%	ABSTRACT
%----------------------------------------------------------------------------------------
%\begin{abstract}
%\noindent O presente trabalho analisa o uso de redes neurais artificiais do
%tipo perceptron de múltiplas camadas para fazer classificação de dados de
%comparação de registros, a fim de categorizá-los como registros correspondentes
%ou não-correspondentes, em um processo conhecido como record linkage.  A base
%de dados contém os resultados de comparação de dados demográficos provenientes
%do registro epidemiológico de câncer do estado alemão de Rhine-Westphalia \cite{west}.
%Os resultados mostram que a rede neural usada neste trabalho conseguiu
%classificar os registros satisfatoriamente.
%
% 
%\end{abstract}

\newpage

\section{Exercícios}

\begin{enumerate}
    \item  \hfill \\
        \begin{enumerate}
            \item[a] - $in(S) = \{v \in S | \forall v, uE\textsuperscript{*} v, u \in V \}$
            \item[b] - $out(S) = \{v \in S | \forall v, vE\textsuperscript{*} u, u \in V \}$
        \end{enumerate}
    \item $G\textsuperscript{-1} = (V, \{(u,v) | (v,u) \in E\})$
    \item $SCC(u) = \{v \in | uE\textsuperscript{*}v \land vE\textsuperscript{*}u\}$
    \item Sim, é possível. De fato, o DFS funciona de forma similar.
    \item $(V, E_{DAG}) = (V, \{r(u),r(v)|uEv\})$
    \item O texto faz referência à propriedades de dígrafos acíclicos e considera que $\forall v \in V, \exists (v,v)$, mesmo os vértices não tendo \textit{loops}. O texto também estabelece uma relação entre vértices em que é possível aplicar uma ordem total para obter a ordenação topológica.
    \item $P(r(u),r(v))$ é o conjunto de vértices intermediários entre $u$ e $v$
    \item \hfill \\
        \begin{algorithmic}
            \Function{InsereNovoVertice}{Grafo, NovaAresta}
                \State $(V,E)\gets Grafo$
                \State $SCSS\gets Tarjan(Grafo)$
                \State $(u,v) \gets NovaAresta$
                \State $AdicionaNovaArestaEmE(NovaAresta)$
                \ForAll{$c1 \in SCCS$}
                    \ForAll{$c2 \in SCCS$}
                        \If {$(u \in c1 \land v \in c2) \lor (v \in c1 \land u \in c2)$}
                            \State $SCSS\gets Tarjan(Grafo)$
                            \State $break$
                        \EndIf
                    \EndFor
                \EndFor
            \EndFunction
        \end{algorithmic}
    \item \hfill \\
        \Function{TransformaGemDAG}{Grafo}
            \Comment{O enunciado desta questão está meio estranho. Voltar nela depois}
        \EndFunction
    \item
\end{enumerate}


\end{document}
