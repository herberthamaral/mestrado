\documentclass[11pt]{article}
\usepackage{epsfig,psfrag}
\usepackage{amssymb}
\usepackage[latin1]{inputenc}
\usepackage[T1]{fontenc}
\usepackage{lipsum} % Package to generate dummy text throughout this template
\usepackage[brazil]{babel}
\usepackage[sc]{mathpazo} % Use the Palatino font
\usepackage[T1]{fontenc} % Use 8-bit encoding that has 256 glyphs
\linespread{1.05} % Line spacing - Palatino needs more space between lines
\usepackage{microtype} % Slightly tweak font spacing for aesthetics

\usepackage[hmarginratio=1:1,top=32mm,columnsep=20pt]{geometry} % Document margins
\usepackage{multicol} % Used for the two-column layout of the document
\usepackage[hang, small,labelfont=bf,up,textfont=it,up]{caption} % Custom captions under/above floats in tables or figures
\usepackage{booktabs} % Horizontal rules in tables
\usepackage{float} % Required for tables and figures in the multi-column environment - they need to be placed in specific locations with the [H] (e.g. \begin{table}[H])
\usepackage{hyperref} % For hyperlinks in the PDF

\usepackage{lettrine} % The lettrine is the first enlarged letter at the beginning of the text
\usepackage{paralist} % Used for the compactitem environment which makes bullet points with less space between them

\usepackage{abstract} % Allows abstract customization
\renewcommand{\abstractnamefont}{\normalfont\bfseries} % Set the "Abstract" text to bold
\renewcommand{\abstracttextfont}{\normalfont\small\itshape} % Set the abstract itself to small italic text

\usepackage{titlesec} % Allows customization of titles

\usepackage{fancyhdr} % Headers and footers
\pagestyle{fancy} % All pages have headers and footers
\fancyhead{} % Blank out the default header
\fancyfoot{} % Blank out the default footer
\fancyhead[C]{{\vspace{-3cm}{\hspace{-4cm}\mbox{\begin{minipage}{1.5cm} \epsfxsize=2cm
\centerline{\epsffile{unimontes.eps}}
\end{minipage}}}{\hspace{.6cm} Intelig�ncia Computacional $\bullet$ 2015}} 
} % Custom header text
\fancyfoot[RO,LE]{\thepage} % Custom footer text




%----------------------------------------------------------------------------------------
%	TITLE SECTION
%----------------------------------------------------------------------------------------

\title{\vspace{.5cm}\fontsize{24pt}{10pt}\selectfont\textbf{\sc T�tulo}} % Article title

\author{
\large
\textsc{Nome}\\[2mm] % Your name
\normalsize Departamento de Ci�ncias da Computa��o \\
\normalsize Universidade Estadual de Montes Claros \\
\normalsize \href{mailto:nome1@sec.com}{nome1@sec.com} % Your email address
\vspace{-5mm}
}
\date{\today}

%----------------------------------------------------------------------------------------

\begin{document}

\maketitle % Insert title

\thispagestyle{fancy} % All pages have headers and footers

%----------------------------------------------------------------------------------------
%	ABSTRACT
%----------------------------------------------------------------------------------------
\selectlanguage{brazil}
\begin{abstract}
\noindent  Escreva um resumo de m�ximo 100 palavras.

\end{abstract}

\section{Introdu��o}

\section{Desenvolvimento}

\section{Resultados}

\section{Considera��es}

\section*{Refer�ncias}

Exemplos de como citar e como inserir refer�ncias

\begin{enumerate}

\item article: \cite{best,pier,jame}

\item proceedings: \cite{Weiss94}

\item inproceedings: \cite{conftypical,Moss94}

\item book: \cite{ArmsRace1}

\item edition: \cite{ComplexVariables}

\item editor: \cite{ColBenh:93}

\item series: \cite{inbook}

\item tech report: \cite{hobby1992,CSTR116}

\item unpublished: \cite{VoQS}

\item phd thesis: \cite{Brown:1988:IEL}

\item masters thesis: \cite{segms}

\item incollection: \cite{Jones:Phoneme}

\item booklet: \cite{URW:kerning}

\item misc: \cite{netpbm}

\begin{thebibliography}{10}

\bibitem{best}
B.~W. Bestbury, {$R$}-matrices and the magic square, {\em J. Phys. A} {\bf
  36}(7)  (2003)  1947--1959.

\bibitem{pier}
P.~X. Deligne and B.~H. Gross, On the exceptional series, and its descendants,
  {\em C. R. Math. Acad. Sci. Paris} {\bf 335}(11)  (2002)  877--881.

\bibitem{jame}
J.~M. Landsberg and L.~Manivel, Triality, exceptional {L}ie algebras and
  {D}eligne dimension formulas, {\em Adv. Math.} {\bf 171}(1)  (2002)  59--85.

\bibitem{Weiss94}
G.~H. Weiss (ed.), {\em Contemporary {P}roblems in {S}tatistical {P}hysics}
  (SIAM, Philadelphia, 1994).

\bibitem{conftypical}
R.~K. Gupta and S.~D. Senturia, Pull-in time dynamics as a measure of absolute
  pressure, in {\em Proc. {IEEE} Int. Workshop on Microelectromechanical
  Systems $(${MEMS}'97$)$\/} (Nagoya, Japan, 1997), pp. 290--294.

\bibitem{Moss94}
F.~Moss, Stochastic resonance: From the ice ages to the monkey's ear, in {\em
  Contemporary {P}roblems in {S}tatistical {P}hysics\/}, ed. G.~H. Weiss (SIAM,
  Philadelphia, 1994), pp. 205--253.

\bibitem{ArmsRace1}
L.~F. Richardson, {\em Arms and {I}nsecurity} (Boxwood, Pittsburg, 1960).

\bibitem{ComplexVariables}
R.~V. Churchill and J.~W. Brown, {\em Complex {V}ariables and {A}pplications},
  5th edn. (McGraw-Hill, 1990).

\bibitem{ColBenh:93}
F.~Benhamou and A.~Colmerauer (eds.), {\em Constraint {L}ogic {P}rogramming,
  {S}elected {R}esearch} (MIT Press, 1993).

\bibitem{inbook}
D.~W. Baker and N.~L. Carter, {\em Seismic {V}elocity {A}nisotropy {C}alculated
  for {U}ltramafic {M}inerals and {A}ggregates}, in {\em Flow and {F}racture of
  {R}ocks\/}, eds. H.~C. Heard, I.~V. Borg, N.~L. Carter and C.~B. Raleigh,
  Geophys. Mono., Vol.~16.
\newblock (Am. Geophys. Union, 1972), pp. 157--166.

\bibitem{hobby1992}
J.~D. Hobby, {A User's Manual for MetaPost}, Tech. Rep. 162, AT\&T Bell
  Laboratories (Murray Hill, New Jersey, 1992).

\bibitem{CSTR116}
B.~W. Kernighan, {PIC}---{A} graphics language for typesetting, Computing
  Science\break Technical Report 116, AT\&T Bell Laboratories (Murray Hill, New
  Jersey, 1984).

\bibitem{VoQS}
M.~J. Ball, J.~Esling and C.~Dickson, {VoQS: Voice Quality Symbols}, Revised to
  1994,  (1994).

\bibitem{Brown:1988:IEL}
M.~E. Brown, {\em An Interactive Environment for Literate Programming}, PhD
  thesis, Texas A\&M University, (TX, USA, 1988), pp. ix + 102.

\bibitem{segms}
G.~S. Lodha, Quantitative interpretation of ariborne electromagnetic response
  for a spherical model, Master's thesis, University of Toronto  (1974).

\bibitem{Jones:Phoneme}
D.~Jones, {The term ``phoneme''}, in {\em Phonetics in Linguistics: A Book of
  Reading\/}, eds. W.~E. Jones and J.~Laver (Longman, London, 1973) pp.
  187--204.

\bibitem{URW:kerning}
B.~Murrey, Kerning on the fly, URW Non-Plus-Ultra Typography series (Pittsburg,
  1991).

\bibitem{netpbm}
B.~Davidsen, Netpbm  (1993),
  \url{ftp://ftp.wustl.edu/graphics/graphics/packages/NetPBM}.

\end{thebibliography}

\end{enumerate}

\end{document}
